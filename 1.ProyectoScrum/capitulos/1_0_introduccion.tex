\chapter{Introducción}

\section{Propósito}
Nuestro sistema se basa en crear un sistema y una aplicación móvil dedicada a ofrecer servicios multifuncionales (arreglos, reparaciones e instalaciones) para los hogares en varios sectores de Lima Metropolitana. Para esto contaremos con personas especializadas en 3 rubros (carpintería, gasfitería y electricista), se podrá hacer una clasificación de nuestros servicios en función a la frecuencia con la que se realicen.
\section{Alcance}
Tal y como se ha indicado, el sistema de servicios domésticos va a ofrecer diversas actividades relacionadas con el mantenimiento y reparaciones dentro de las viviendas, donde se pueden contemplar diversos rubros de servicios como son el de electricista, carpintería y gasfitería.
Entre los distintos tipos de instalaciones se encuentran:
-Instalaciones eléctricas, calefacción, aire acondicionado, etc.
-Instalación de puertas, muebles.
-Reparación y mantenimiento de cocinas y electrodomésticos.


\section{Definiciones, siglas y abreviaturas}
-Django:
Es el framework que nos permitirá diseñar el sistema en la página web , realizar las aplicaciones de cualquier complejidad a corto plazo.
-Springboot
Spring boot nos permitirá centrar y simplificar el proceso de nuestro sistema.
-Android
Android es el sistema operativo gratuito y con una plataforma muy amplia , lo usaremos para la implementación en el móvil.


\section{Referencias}
-https://openwebinars.net/blog/que-es-django-y-por-que-usarlo/
-https://www.xatakandroid.com/sistema-operativo/que-es-android
-https://www.arquitecturajava.com/que-es-spring-boot/
